\documentclass[a4paper, 12pt]{article}

% Page layout settings
\usepackage[a4paper, left=2cm, right=1cm, top=2cm, bottom=1cm]{geometry}
\usepackage{minted}
\usepackage{afterpage}
\usepackage{tikz}
\usepackage[framemethod=TikZ]{mdframed}
\usepackage{graphicx} % for image inclusion
\usepackage{float} % for positioning images
\usepackage{array}
\usepackage{fancyhdr}  % Load fancyhdr package
\renewcommand{\headrulewidth}{0pt}  % Remove the horizontal line in header
\pagestyle{empty}  % Start with an empty default style

\pagestyle{fancy}  
\fancyhf{}  
\fancyhead[L]{Date: }  % Increase font size & bold
\fancyhead[R]{Page No: } 

\pagenumbering{gobble}

\begin{document}

% EXPERIMENT 10
\newpage
\begin{center}
\section*{\LARGE \textbf{\underline{Process Management System Calls - 01}}} 
\end{center}

\subsection*{\underline{AIM}}
\begin{quote}
%<aim>%
Create a new process using the fork() system call. Have the processes introduce themselves with their process IDs as follows:

Parent: I am \textless myID\textgreater, and my child is \textless mychildID\textgreater \\
Child: I am \textless myID\textgreater, and my parent is \textless myparentID\textgreater

Use the pstree command to display the process tree for the child process starting from the
init process. The child process is to be “highlighted” in the tree.
\end{quote}

\subsection*{\underline{ALGORITHM}}
\begin{quote}
%<algorithm>%
1. Start\\
2. Invoke the "fork()" system call and store the return value into "pid"\\
3. If pid is less than zero, goto step-6\\
4. Else if pid is zero,\\
\hspace*{1cm} 4.1. Invoke the "getpid()" system call and output the pid of the current process \hspace*{2cm} (child pid)\\
\hspace*{1cm} 4.2. Invoke the "getppid()" system call and output the pid of the parent process \hspace*{2cm}  (parent pid) \\
5. Else,\\
\hspace*{1cm} 5.1. Output "pid" (child pid)\\
\hspace*{1cm} 5.2. Invoke the "getpid()" system call and output pid of the current process (parent \hspace*{2cm} pid)\\
6. Execute the command "pstree -p $|$ grep $<childId>$" in the command line\\
7. Stop\\
\end{quote}

\subsection*{\underline{PROGRAM}}
\begin{quote}
%<code>%
\inputminted[fontsize=\small,breaklines,breakanywhere]{c}{fork_process_demo.c}%program name
\end{quote}

% OUTPUT
\newpage
\subsection*{\underline{OUTPUT}}

\begin{figure}[H] 
    \centering
    \fbox{
        \begin{minipage}{0.8\textwidth} % Adjusts width dynamically
            \centering
            \includegraphics[width=\textwidth]{10.png} 
        \end{minipage}
    }
\end{figure}

% RESULT

\subsection*{\underline{RESULT}}
\begin{quote}
%<result>%
C program demonstrating the use of fork system call was executed and verified.
\end{quote}

% EXPERIMENT 11
\newpage
\begin{center}
\section*{\LARGE \textbf{\underline{Process Management System Calls - 02}}} 
\end{center}

\subsection*{\underline{AIM}}
\begin{quote}
%<aim>%
Fork two processes. The parent process inputs an integer array. One child prints all the
even numbers in the array and the other prints all the odd numbers. Use waitpid() system
call to ensure that all odd numbers are printed first followed by the even numbers.
\end{quote}

\subsection*{\underline{ALGORITHM}}
\begin{quote}
%<algorithm>%
1. Start\\
2. Read number of elements, "n"\\
3. Read the elements of the array, "arr"\\
4. Invoke the "fork()" system call and store the return value into "pid1"\\
5. If "pid1" is less than zero, goto step-8\\
6. Else if "pid1" is zero, loop through elements of "arr" and output the odd elements\\
7. Else,\\
\hspace*{1cm} 7.1. Invoke the "waitpid()" system call to wait until child process terminates\\
\hspace*{1cm} 7.2. Invoke the "fork()" system call and store the return value into "pid2"\\
\hspace*{1cm} 7.3. If "pid2" is less than zero goto step-8\\
\hspace*{1cm} 7.4. Else if "pid2" is zero, loop through elements of "arr" and output the even \hspace*{2cm} elements\\
\hspace*{1cm} 7.5. Else, invoke the "waitpid()" system call to wait until the child process \hspace*{2cm} terminates\\
8. Stop\\
\end{quote}

\subsection*{\underline{PROGRAM}}
\begin{quote}
%<code>%
\inputminted[fontsize=\small,breaklines,breakanywhere]{c}{odd_even_processor.c}%program name
\end{quote}

% OUTPUT
\newpage
\subsection*{\underline{OUTPUT}}

\begin{figure}[H] 
    \centering
    \fbox{
        \begin{minipage}{300} % Adjusts width dynamically
            \centering
            \includegraphics[width=300]{11.png} 
        \end{minipage}
    }
\end{figure}

% RESULT

\subsection*{\underline{RESULT}}
\begin{quote}
%<result>%
C program to display odd and even numbers from input using processes was executed and verified.
\end{quote}

% EXPERIMENT 12
\newpage
\begin{center}
\section*{\LARGE \textbf{\underline{CPU Scheduling Algorithms}}} 
\end{center}

\subsection*{\underline{AIM}}
\begin{quote}
%<aim>%
Input a list of processes, their CPU burst times (integral values), arrival times, and
priorities. Then simulate the following scheduling algorithms on the process mix:\\
(a) FCFS\\
(b) SJF\\
(c) SRTF\\
(d) non-preemptive priority (a larger priority number implies a higher priority)\\
(e) RR (with a quantum of 3 units)\\
Determine which algorithm results in the minimum average waiting time.
\end{quote}

\subsection*{\underline{ALGORITHM}}
\begin{quote}
%<algorithm>%
1. Start\\
2. Define function for FCFS scheduling\\
\hspace*{1cm} 2.1. Initialize "current\_time" and "total\_waiting" to 0\\
\hspace*{1cm} 2.2. For each process, repeat steps 2.3-2.6 until end of process list\\
\hspace*{1cm} 2.3. Compute completion time\\
\hspace*{1cm} 2.4. Compute turnaround time\\
\hspace*{1cm} 2.5. Compute waiting time and add to "total\_waiting"\\
\hspace*{1cm} 2.6. Update "current\_time"\\
\hspace*{1cm} 2.7. Output average waiting time\\
3. Define function for SJF scheduling\\
\hspace*{1cm} 3.1. Initialize "completed", "total\_waiting", and "current\_time" to 0\\
\hspace*{1cm} 3.2. Track completion status with "is\_completed"\\
\hspace*{1cm} 3.3. While not all processes are completed, repeat steps 3.4-3.8 until all processes are completed\\
\hspace*{1cm} 3.4. Select the shortest available process\\
\hspace*{1cm} 3.5. Compute completion, turnaround, and waiting times\\
\hspace*{1cm} 3.6. Add waiting time to "total\_waiting"\\
\hspace*{1cm} 3.7. Mark process as completed\\
\hspace*{1cm} 3.8. Update "current\_time"\\
\hspace*{1cm} 3.9. Output average waiting time\\
4. Define function for priority scheduling\\
\hspace*{1cm} 4.1. Initialize "completed", "total\_waiting", and "current\_time" to 0\\
\hspace*{1cm} 4.2. Track completion status with "is\_completed"\\
\hspace*{1cm} 4.3. While not all processes are completed, repeat steps 4.4-4.8 until all processes are completed\\
\hspace*{1cm} 4.4. Select highest-priority available process\\
\hspace*{1cm} 4.5. Compute completion, turnaround, and waiting times\\
\hspace*{1cm} 4.6. Add waiting time to "total\_waiting"\\
\hspace*{1cm} 4.7. Mark process as completed\\
\hspace*{1cm} 4.8. Update "current\_time"\\
\hspace*{1cm} 4.9. Output average waiting time\\
5. Define function for round robin scheduling\\
\hspace*{1cm} 5.1. Initialize "current\_time", "completed", and "total\_waiting" to 0\\
\hspace*{1cm} 5.2. Track remaining burst time with "remaining\_time"\\
\hspace*{1cm} 5.3. While not all processes are completed, repeat steps 5.4-5.8 until all processes are completed\\
\hspace*{1cm} 5.4. Select a process with remaining burst time\\
\hspace*{1cm} 5.5. Execute for a time slice or remaining time\\
\hspace*{1cm} 5.6. Update "current\_time" and "remaining\_time"\\
\hspace*{1cm} 5.7. If process completes, compute turnaround and waiting times\\
\hspace*{1cm} 5.8. Update "total\_waiting"\\
\hspace*{1cm} 5.9. Output average waiting time\\
6. Define function for SRTF scheduling\\
\hspace*{1cm} 6.1. Initialize "current\_time", "completed", and "total\_waiting" to 0\\
\hspace*{1cm} 6.2. Track remaining burst time with "remaining\_time"\\
\hspace*{1cm} 6.3. While not all processes are completed, repeat steps 6.4-6.8 until all processes are completed\\
\hspace*{1cm} 6.4. Select process with shortest remaining time\\
\hspace*{1cm} 6.5. Execute for 1 time unit\\
\hspace*{1cm} 6.6. Update "current\_time" and remaining time\\
\hspace*{1cm} 6.7. If process completes, compute turnaround and waiting times\\
\hspace*{1cm} 6.8. Update "total\_waiting"\\
\hspace*{1cm} 6.9. Output average waiting time\\
7. Read number of processes, $n$\\
8. Read process list (ID, Arrival Time, Burst Time, Priority)\\
9. Sort process list by arrival time\\
10. Compute average waiting times using all scheduling algorithms\\
11. Output best scheduling algorithm with lowest average waiting time\\
12. Stop\\

\end{quote}

\subsection*{\underline{PROGRAM}}
\begin{quote}
%<code>%
\inputminted[fontsize=\small,breaklines,breakanywhere]{c}{cpu_scheduling.c}%program name
\end{quote}

% OUTPUT
\newpage
\subsection*{\underline{OUTPUT}}

\begin{figure}[H] 
    \centering
    \fbox{
        \begin{minipage}{300} % Adjusts width dynamically
            \centering
            \includegraphics[width=300]{12.png} 
        \end{minipage}
    }
\end{figure}

% RESULT

\subsection*{\underline{RESULT}}
\begin{quote}
%<result>%
C program simulating various scheduling algorithms on a process mix was executed and verified.
\end{quote}

% EXPERIMENT 13
\newpage
\begin{center}
\section*{\LARGE \textbf{\underline{Banker's Algorithm}}} 
\end{center}

\subsection*{\underline{AIM}}
\begin{quote}
%<aim>%
Obtain a (deadlock-free) process mix and simulate the banker’s algorithm to determine
a safe execution sequence.
\end{quote}

\subsection*{\underline{ALGORITHM}}
\begin{quote}
%<algorithm>%
1. Start\\
2. Read the "available" vector\\
3. Read the number of processes, "np" and number of distinct resource types, "nr"\\
4. Read the "allocated" and "max" matrices for each process from the user\\
5. Compute the "need" matrix subtracting the allocation from the max\\
6. Initialize vector "completed" (having as many elements as the number of processes) \hspace*{0.4cm} to false\\
7. Initialize vector "work" (having as many elements as the number of distinct resource \hspace*{0.4cm} types) to available\\
8. Initialize vector "sequence" (having as many elements as the number of processes) to \hspace*{0.4cm} store the safe sequence\\
9. Search "completed" for an element whose value is false having need less than work\\
10. If no such element exists, goto step-14\\
11. Add the "allocated" vector of the process to the "work" vector\\
12. Update the completed flag of the process to true\\
13. Append the process id/ name into "sequence" and goto step-9\\
14. If all entries of the "completed" vector are true, system is in safe state, goto step-16\\
15. Else, system is in unsafe state, goto step-17 \\
16. Output the safe sequence\\
17. Stop\\
\end{quote}

\subsection*{\underline{PROGRAM}}
\begin{quote}
%<code>%
\inputminted[fontsize=\small,breaklines,breakanywhere]{c}{bankers_algorithm.c}%program name
\end{quote}

% OUTPUT
\subsection*{\underline{OUTPUT}}

\begin{figure}[H] 
    \centering
    \fbox{
        \begin{minipage}{300} % Adjusts width dynamically
            \centering
            \includegraphics[width=300]{13.png} 
        \end{minipage}
    }
\end{figure}

% RESULT

\subsection*{\underline{RESULT}}
\begin{quote}
%<result>%
C program simulating banker’s algorithm on a process mix was executed and verified.
\end{quote}

\end{document}