\documentclass[a4paper, 12pt]{article}

% Page layout settings
\usepackage[a4paper, left=2cm, right=1cm, top=2cm, bottom=1cm]{geometry}
\usepackage{minted}
\usepackage{afterpage}
\usepackage{tikz}
\usepackage[framemethod=TikZ]{mdframed}
\usepackage{graphicx} % for image inclusion
\usepackage{float} % for positioning images
\usepackage{array}
\usepackage{fancyhdr}  % Load fancyhdr package
\renewcommand{\headrulewidth}{0pt}  % Remove the horizontal line in header
\pagestyle{empty}  % Start with an empty default style

\pagestyle{fancy}  
\fancyhf{}  
\fancyhead[L]{Date: }  % Increase font size & bold
\fancyhead[R]{Page No: } 

\pagenumbering{gobble}

\begin{document}

% EXPERIMENT 2
\newpage
\begin{center}
\section*{\LARGE \textbf{\underline{Shell Scripting Experiment 1}}} % replace with relevant experiment number (shell scripting experiments 2 - 9)
\end{center}

\subsection*{\underline{AIM}}
\begin{quote}
%<aim>%
Create a menu driven script to print the following
processor information using the proc file system:\\
(a) vendor (manufacturer) id\\
(b) model name\\
(c) processor generation\\
(d) number of processor chips\\
(e) number of processor cores\\
(f) is your processor hyperthreaded?\\
(g) number of logical processors\\
(h) id of the core to which each logical processor is mapped\\
(i) speed of each logical processor\\
(j) cache size\\
\end{quote}

\subsection*{\underline{ALGORITHM}}
\begin{quote}
%<algorithm>%
1. Start\\
2. Display the menu\\
3. Prompt the user to enter a choice\\
4. Based on the user's choice display the relevant information\\
5. Loop until the user selects the exit option\\
6. End
\end{quote}

\subsection*{\underline{PROGRAM}}
\begin{quote}
%<code>%
\inputminted[fontsize=\small,breaklines,breakanywhere]{sh}{shell1.sh}%program name
\end{quote}

% OUTPUT
\newpage
\subsection*{\underline{OUTPUT}}

\begin{figure}[H] 
    \centering
    \fbox{
        \begin{minipage}{0.8\textwidth} % Adjusts width dynamically
            \centering
            \includegraphics[width=\textwidth]{2-1.png} % Ensures it fits within the page
            \includegraphics[width=\textwidth]{2-2.png}
        \end{minipage}
    }
\end{figure}

\begin{figure}[H] 
    \centering
    \fbox{
        \begin{minipage}{0.8\textwidth} % Adjusts width dynamically
            \centering
            \includegraphics[width=\textwidth]{2-3.png} % Ensures it fits within the page
        \end{minipage}
    }
\end{figure}

% RESULT

\subsection*{\underline{RESULT}}
\begin{quote}
%<result>%
A bash script for displaying processor information was executed and verified.
\end{quote}

% EXPERIMENT 3
\newpage
\begin{center}
\section*{\LARGE \textbf{\underline{Shell Scripting Experiment 2}}} % replace with relevant experiment number (shell scripting experiments 2 - 9)
\end{center}

\subsection*{\underline{AIM}}
\begin{quote}
%<aim>%
Create a menu driven script to print the following memory information using the proc file
system:\\
(a) total memory in the system\\
(b) amount of memory left unused\\
(c) amount of memory available\\
(d) amount of memory used as cache.

\end{quote}

\subsection*{\underline{ALGORITHM}}
\begin{quote}
%<algorithm>%
1. Start\\
2. Display the menu.\\
3. Get the option as user input.\\
4. Based on the user's choice, fetch and display the required information from '/proc/meminfo' file.\\
5. Loop until the user selects the exit option.\\
6. End\\
\end{quote}

\subsection*{\underline{PROGRAM}}
\begin{quote}
%<code>%
\inputminted[fontsize=\small,breaklines,breakanywhere]{sh}{shell1.sh}%program name
\end{quote}

% OUTPUT

\subsection*{\underline{OUTPUT}}

\begin{figure}[H] 
    \centering
    \fbox{
        \begin{minipage}{0.8\textwidth} % Adjusts width dynamically
            \centering
            \includegraphics[width=\textwidth]{3.png}  % Ensures it fits within the page
        \end{minipage}
    }
\end{figure}

% RESULT

\subsection*{\underline{RESULT}}
\begin{quote}
%<result>%
A bash script for displaying memory information was executed and verified.
\end{quote}

% EXPERIMENT 4
\newpage
\begin{center}
\section*{\LARGE \textbf{\underline{Shell Scripting Experiment 3}}} % replace with relevant experiment number (shell scripting experiments 2 - 9)
\end{center}

\subsection*{\underline{AIM}}
\begin{quote}
%<aim>%
Implement a menu driven scientific calculator. You should consider arithmetic operators,
sin,cos, tan, power and square root.

\end{quote}

\subsection*{\underline{ALGORITHM}}
\begin{quote}
%<algorithm>%
1. Start\\
2. Display Menu\\
3. Get user input - the operation to be performed.\\
4. For operations - Add, Subtract, Multiply and Divide- read two operands from the user.\\
5. For operation "Square" - get the number to be squared the user.\\
6. For operation "Exponentiation" - read the base, exponent pair from the user.\\
7. For trigonometric operations read the angle in degrees as input and convert and store the value in radians.\\
8. Use the read inputs and compute the required results using 'bc -l'.\\
9. Print the result.\\
10. Loop the menu until the 'exit' condition is met.\\
11. End\\
\end{quote}

\subsection*{\underline{PROGRAM}}
\begin{quote}
%<code>%
\inputminted[fontsize=\small,breaklines,breakanywhere]{sh}{shell1.sh}%program name
\end{quote}

% OUTPUT
\newpage

\subsection*{\underline{OUTPUT}}

\begin{figure}[H] 
    \centering
    \fbox{
        \begin{minipage}{0.8\textwidth} % Adjusts width dynamically
            \centering
            \includegraphics[width=\textwidth]{4-1.png} % Ensures it fits within the page
            \includegraphics[width=\textwidth]{4-2.png} % Ensures it fits within the page
        \end{minipage}
    }
\end{figure}

% RESULT

\subsection*{\underline{RESULT}}
\begin{quote}
%<result>%
A bash script for a scientific calculator was executed and verified.
\end{quote}

% EXPERIMENT 5
\newpage
\begin{center}
\section*{\LARGE \textbf{\underline{Shell Scripting Experiment 4}}} % replace with relevant experiment number (shell scripting experiments 2 - 9)
\end{center}

\subsection*{\underline{AIM}}
\begin{quote}
%<aim>%
Print all prime numbers in a range.

\end{quote}

\subsection*{\underline{ALGORITHM}}
\begin{quote}
%<algorithm>%
1. Start\\
2. Read the range 'start' and 'end'\\
3. Loop till step 8 from 'start' to 'end'\\
4. If the current number is less than 2, skip.\\
5. If the current number is 2, print.\\
6. Else - check for remainders on division from current number to square root of current number.\\
7. If no remainders exist - skip.\\
8. Else - print current number.\\
9. End\\
\end{quote}

\subsection*{\underline{PROGRAM}}
\begin{quote}
%<code>%
\inputminted[fontsize=\small,breaklines,breakanywhere]{sh}{shell1.sh}%program name
\end{quote}

% OUTPUT

\subsection*{\underline{OUTPUT}}

\begin{figure}[H] 
    \centering
    \fbox{
        \begin{minipage}{0.8\textwidth} % Adjusts width dynamically
            \centering
            \includegraphics[width=\textwidth]{5.png} % Ensures it fits within the page
        \end{minipage}
    }
\end{figure}

% RESULT

\subsection*{\underline{RESULT}}
\begin{quote}
%<result>%
A bash script to print prime numbers in a given range was executed and verified.
\end{quote}

% EXPERIMENT 6
\newpage
\begin{center}
\section*{\LARGE \textbf{\underline{Shell Scripting Experiment 5}}} % replace with relevant experiment number (shell scripting experiments 2 - 9)
\end{center}

\subsection*{\underline{AIM}}
\begin{quote}
%<aim>%
Write a menu driven program to convert a positive decimal number to binary, octal or
hexadecimal based on the user's choice. Each of the conversions should be implemented
using separate functions
\end{quote}

\subsection*{\underline{ALGORITHM}}
\begin{quote}
%<algorithm>%
1. Start\\
2. Prompt the user to enter a number\\
3. Validate the input is a positive decimal number\\
4. Display the menu\\
5. Prompt the user to enter a choice\\
6. If choice is 1:\\
   Convert to binary and display the result\\
7. If choice is 2:\\
   Convert to octal and display the result\\
8. If choice is 3:\\
   Convert to hexadecimal and display the result\\
9. If choice is 4:\\
   Display exit message\\
10. End\\
\end{quote}

\subsection*{\underline{PROGRAM}}
\begin{quote}
%<code>%
\inputminted[fontsize=\small,breaklines,breakanywhere]{sh}{shell1.sh}%program name
\end{quote}

% OUTPUT

\subsection*{\underline{OUTPUT}}

\begin{figure}[H] 
    \centering
    \fbox{
        \begin{minipage}{0.8\textwidth} % Adjusts width dynamically
            \centering
            \includegraphics[width=\textwidth]{6.png} 
            % Ensures it fits within the page
        \end{minipage}
    }
\end{figure}

% RESULT

\subsection*{\underline{RESULT}}
\begin{quote}
%<result>%
A menu-driven bash script for number conversion was executed and verified.
\end{quote}

% EXPERIMENT 7
\newpage
\begin{center}
\section*{\LARGE \textbf{\underline{Shell Scripting Experiment 6}}} % replace with relevant experiment number (shell scripting experiments 2 - 9)
\end{center}

\subsection*{\underline{AIM}}
\begin{quote}
%<aim>%
Read an integer N, a string word and a filename via command line. If a word occurs more
than N times in the file, remove all its occurrences in the file.

\end{quote}

\subsection*{\underline{ALGORITHM}}
\begin{quote}
%<algorithm>%
1. Start\\
2. Prompt the user to enter the number of occurrences (N)\\
3. Prompt the user to enter the word to search for\\
4. Prompt the user to enter the filename\\
5. Check if the file exists\\
   - If the file does not exist:\\
     - Display error message\\
     - Exit program\\
6. Count the occurrences of the word in the file\\
7. Check if the word occurs more than N times\\
   - If word count > N:\\
     - Remove all occurrences of the word from the file\\
     - Display message that all occurrences have been removed\\
   - If word count <= N:\\
     - Display message with the count and that it's not more than N\\
8. End\\
\end{quote}

\subsection*{\underline{PROGRAM}}
\begin{quote}
%<code>%
\inputminted[fontsize=\small,breaklines,breakanywhere]{sh}{shell1.sh}%program name
\end{quote}

% OUTPUT

\subsection*{\underline{OUTPUT}}

\begin{figure}[H] 
    \centering
    \fbox{
        \begin{minipage}{0.8\textwidth} % Adjusts width dynamically
            \centering
            \includegraphics[width=\textwidth]{7.png} 
            % Ensures it fits within the page
        \end{minipage}
    }
\end{figure}

% RESULT

\subsection*{\underline{RESULT}}
\begin{quote}
%<result>%
A bash script for removing occurrences of a word beyond N times was executed and verified.
\end{quote}

% EXPERIMENT 8
\newpage
\begin{center}
\section*{\LARGE \textbf{\underline{Shell Scripting Experiment 7}}} % replace with relevant experiment number (shell scripting experiments 2 - 9)
\end{center}

\subsection*{\underline{AIM}}
\begin{quote}
%<aim>%
You have a file s4csb2.txt with the following format:
AdmnNo,Name,Address,PhoneNo,Email.
Now create a command recedit which is to be invoked as:
recedit op ID
Here op is one among the operations add,search,update,delete and ID is an admission
number.\\
The command behaviour is summarized below:\\
• raise an error if a sufficient number of arguments are not supplied.\\
• to add a new student with the given ID, get the other pertinent details and create a new row for the student in the file s4csb1.txt. If such a student already exists, raise an error.\\
• if the operation is search, the pertinent details should be printed if such a student exists or
raise an error otherwise.\\
• if the operation is updated, get the details and update the student record.\\
• if the operation is delete, remove the student record.
\end{quote}

\subsection*{\underline{ALGORITHM}}
\begin{quote}
%<algorithm>%
1. Start\\
2. Define file name as "s4csb2.txt"\\
3. Check if the number of arguments is not equal to 2\\
   - If true:\\
     - Display error message\\
     - Display usage information\\
     - Exit program with error code 1\\
4. Set operation to first argument\\
5. Set student ID to second argument\\
6. Define function to check if student exists in file\\
7. Based on the operation value:\\
   - If operation is "add":\\
     - Check if student exists\\
       - If true: Display "Exists" message\\
       - If false:\\
         - Prompt for details\\
         - Append student record to file\\
         - Display success message\\
   - If operation is "search":\\
     - Check if student exists\\
       - If true: Display the student record\\
       - If false: Display "not present" message\\
   - If operation is "update":\\
     - Check if student exists\\
       - If true:\\
         - Prompt for new details\\
         - Replace existing record with updated information\\
         - Display success message\\
       - If false: Display error message\\
   - If operation is "delete":\\
     - Check if student exists\\
       - If true:\\
         - Remove student record from file\\
         - Display success message\\
       - If false: Display error message\\
8. End\\
\end{quote}

\subsection*{\underline{PROGRAM}}
\begin{quote}
%<code>%
\inputminted[fontsize=\small,breaklines,breakanywhere]{sh}{shell1.sh}%program name
\end{quote}

% OUTPUT
\newpage

\subsection*{\underline{OUTPUT}}

\begin{figure}[H] 
    \centering
    \fbox{
        \begin{minipage}{0.8\textwidth} % Adjusts width dynamically
            \centering
            \includegraphics[width=\textwidth]{8.png} 
            % Ensures it fits within the page
        \end{minipage}
    }
\end{figure}

% RESULT

\subsection*{\underline{RESULT}}
\begin{quote}
%<result>%
A bash script implementing student record management was executed and verified.
\end{quote}

% EXPERIMENT 9
\newpage
\begin{center}
\section*{\LARGE \textbf{\underline{Shell Scripting Experiment 8}}} % replace with relevant experiment number (shell scripting experiments 2 - 9)
\end{center}

\subsection*{\underline{AIM}}
\begin{quote}
%<aim>%
Create a menu driven script to print the following process information:\\
(a) Number of processes forked since the last boot\\
(b) Number of processes currently in the system\\
(c) Number of running processes\\
(d) Number of blocked processes\\
(e) PID of the current shell\\
(f) Number of context switches performed by this shell.\\
How many of these were forcibly taken?
\end{quote}

\subsection*{\underline{ALGORITHM}}
\begin{quote}
%<algorithm>%
1. Start\\
2. Display menu with process information options\\
3. Define function to get number of processes forked since last boot:\\
   - Read value of process since last boot\\
4. Define function to get number of processes currently in system:\\
   - Count all processes using ps command and wc\\
5. Define function to get number of running processes:\\
   - Count processes with state 'R' using ps and grep\\
6. Define function to get number of blocked processes:\\
   - Count processes with state 'D' using ps and grep\\
7. Define function to get PID of current shell:\\
   - Return value of \$\$ variable\\
8. Define function to get number of context switches by this shell:\\
   - Calculate sum of voluntary and involuntary context switches\\
9. Define function to get number of forced context switches by this shell:\\
   - Read involuntary context switches\\
10. Select the option from the menu\\
11. Goto the appropriate function\\
12. End\\
\end{quote}

\subsection*{\underline{PROGRAM}}
\begin{quote}
%<code>%
\inputminted[fontsize=\small,breaklines,breakanywhere]{sh}{shell1.sh}%program name
\end{quote}

% OUTPUT
\newpage

\subsection*{\underline{OUTPUT}}

\begin{figure}[H] 
    \centering
    \fbox{
        \begin{minipage}{0.8\textwidth} % Adjusts width dynamically
            \centering
            \includegraphics[width=\textwidth]{9.png} 
            % Ensures it fits within the page
        \end{minipage}
    }
\end{figure}

% RESULT

\subsection*{\underline{RESULT}}
\begin{quote}
%<result>%
A bash script for displaying process information was executed and verified.
\end{quote}

\end{document}